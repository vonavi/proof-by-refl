\documentclass{beamer}
\usetheme{metropolis}

\usepackage[utf8]{inputenc}
\usepackage[english]{babel}
\usepackage{amssymb,amsmath}

\usepackage{hyperref}
\hypersetup{colorlinks,linkcolor=}

\usepackage{minted}
\usemintedstyle{tango}

\usepackage{epigraph}
\usepackage{ragged2e}

\title{Proof by Reflection with Examples}
\author{Vladimir Ivanov}
\date{\today}

\begin{document}
\begin{frame}[plain,noframenumbering]
  \maketitle
\end{frame}

\begin{frame}
  \section{Proof by Reflection}
  \setlength{\epigraphwidth}{\textwidth}
  \epigraph{\justifying
    The term \textit{reflection} applies because we will need to translate Gallina propositions into values of inductive types representing syntax, so that Gallina programs may analyze them, and translating such a term back to the original form is called \textit{reflecting} it.
  }{Adam Chlipala,  \href{http://adam.chlipala.net/cpdt/html/Reflection.html}{Library Reflection}}
\end{frame}

\begin{frame}[fragile]
  \frametitle{Evenness Checking}
  Prove that particular natural number constants are even.
\begin{minted}{coq}
Inductive is_even : nat -> Prop :=
| even_O : is_even O
| even_SS : forall x, is_even x -> is_even (S (S x)).

Theorem is_even_two : is_even 2.
Proof. repeat constructor. Qed.

Print is_even_two.
(* is_even_two = even_SS 0 even_O *)
\end{minted}
\end{frame}

\begin{frame}[fragile]
  \frametitle{When Is Proof Slow?}
\begin{minted}{coq}
Goal is_even 9002.
  Time repeat constructor.
  (* 5.703 secs *)
  Show Proof.
  (* even_SS 9000 (even_SS 8998 ... ) *)
Qed.

Goal is_even (10*10*10*10*10*10*10*10*10).
  cbv.
  (* Stack overflow *)
Abort.
\end{minted}
\end{frame}

\begin{frame}[fragile]
  \frametitle{Computational Reflection}
  Expression language including \texttt{nat} literals and multiplication nodes.
\begin{minted}{coq}
Inductive expr :=
| NatO : expr
| NatS (x : expr) : expr
| NatMul (x y : expr) : expr.
\end{minted}
  Tell what number our expression represents:
\begin{minted}{coq}
Fixpoint interp (t : expr) : nat :=
  match t with
  | NatO => O
  | NatS x => S (interp x)
  | NatMul x y => interp x * interp y
  end.
\end{minted}
\end{frame}

\begin{frame}[fragile]
  \frametitle{Goal Reification}
  Reify the goal into abstract syntactic representation:
\begin{minted}{coq}
Ltac reify term :=
  lazymatch term with
  | O => NatO
  | S ?x => let rx := reify x in constr:(NatS rx)
  | ?x * ?y => let rx := reify x in
               let ry := reify y in
               constr:(NatMul rx ry)
  end.
\end{minted}
\end{frame}

\begin{frame}[fragile]
\frametitle{Evenness Checker}
\begin{minted}{coq}
Fixpoint check_is_even_expr (t : expr) : bool :=
  match t with
  | NatO => true
  | NatS x => negb (check_is_even_expr x)
  | NatMul x y =>
    orb (check_is_even_expr x) (check_is_even_expr y)
  end.
\end{minted}
  Whenever the checker returns \texttt{true}, the represented number is even.
\begin{minted}{coq}
Theorem check_is_even_expr_sound (e : expr) :
  check_is_even_expr e = true -> is_even (interp e).
\end{minted}
\end{frame}

\begin{frame}[fragile]
  \frametitle{Reflective Automation}
  Run the algorithm on the reified syntax:
\begin{minted}{coq}
Goal is_even (10*10*10*10*10*10*10*10*10).
  match goal with
  | [ |- is_even ?v ]
    => let e := reify v in
       refine (check_is_even_expr_sound e _)
  end.
  vm_compute. reflexivity.
Qed.
\end{minted}
\end{frame}

\begin{frame}
  \section{Monoid}
\end{frame}

\begin{frame}[fragile]
\begin{minted}{coq}
Lemma trial : forall a b c d,
    (a + b) + (c + d) = a + b + c + d.
Proof.
  intros. repeat rewrite Nat.add_assoc. reflexivity.
Qed.
\end{minted}
\end{frame}

\begin{frame}[fragile]
  \frametitle{Expression Interpretation}
\begin{minted}{coq}
Inductive expr :=
| Var : nat -> expr
| Plus : expr -> expr -> expr.

Fixpoint interp e :=
  match e with
  | Var x => x
  | Plus a b => interp a + interp b
  end.
\end{minted}
\end{frame}

\begin{frame}[fragile]
  \frametitle{Goal Reification}
\begin{minted}{coq}
Ltac reify e :=
  match e with
  | ?e1 + ?e2 => let r1 := reify e1 in
                 let r2 := reify e2 in
                 constr:(Plus r1 r2)
  | _ => constr:(Var e)
  end.

Ltac matac :=
  match goal with
  | [ |- ?me1 = ?me2 ] => let r1 := reify me1 in
                          let r2 := reify me2 in
                          change (interp r1 = interp r2)
  end.
\end{minted}
\end{frame}

\begin{frame}[fragile]
  \frametitle{Reflective Automation}
  \vspace{-.5em}
\begin{minted}{coq}
Fixpoint flatten e :=
  match e with
  | Var x => x :: nil
  | Plus a b => flatten a ++ flatten b
  end.

Fixpoint sum_list l :=
  match l with
  | nil => 0
  | x :: r => x + sum_list r
  end.

Theorem main : forall e1 e2,
    sum_list (flatten e1) = sum_list (flatten e2)
    -> interp e1 = interp e2.
\end{minted}
\end{frame}

\begin{frame}[fragile]
\begin{minted}{coq}
Lemma trial_refl : forall a b c d,
    (a + b) + (c + d) = a + b + c + d.
Proof.
  intros. matac. apply main. simpl. reflexivity.
Qed.
\end{minted}
\end{frame}

\begin{frame}\justifying
  \frametitle{Commutative Monoid}
  We base our overview on the problem of proving equality in a commutative monoid. For example, consider proving the following problem instance, where $\oplus$ is the plus operator in the monoid.
  \begin{align*}
    x \oplus 2 \oplus 3 \oplus 4 = 4 \oplus 3 \oplus 2 \oplus x
  \end{align*}
  We can now write a procedure (\texttt{Mcheck}) that determines whether the terms are equal by flattening each expression into a list and checking whether one list is a permutation of the other.
\end{frame}

\begin{frame}\justifying
  \section{Quantifier Elimination}
  Predicate logic is undecidable in the general case; However, a number of specific theories within predicate logic are in fact decidable. Their decidability is shown through \textit{quantifier elimination}. The idea behind quantifier elimination is to devise a method to transform any given proposition quantified with $\forall$ or $\exists$ into an equivalent one without quantifiers.
\end{frame}

\begin{frame}\justifying
  \frametitle{Theory of Successor on the Natural Numbers}
  Atomic formulae are equalities between terms, each of which is the application of the successor function $S$ some (known) number of times to either a variable or zero. For example:
  \begin{align*}
    S(S(S(x))) = S(S(S(S(O))))
  \end{align*}
  or, more concisely:
  \begin{align*}
    S^3(x) = S^4(O).
  \end{align*}
  Valid atomic formulas: $3 + x = 4$, $3 + x = 7 + y$.
\end{frame}

\begin{frame}
  \frametitle{Decision Procedure}
  Propositions demonstrating how decision procedure works:
  \begin{align*}
    \exists x. \exists y. &
    \\
    & 3 + x = 1 + y \land 8 = 4 + y
    \\
    \iff &
    \\
    & x = 2 \land y = 4
  \end{align*}
\end{frame}

\begin{frame}[fragile]
\begin{minted}{coq}
Inductive base : Set :=
| Zero : base
| Var : nat -> base.

Inductive term : Set := Succ : nat -> base -> term.
\end{minted}

\begin{minted}{coq}
Definition base2nat (l : list nat) (b : base) : nat :=
  match b with
  | Zero => 0
  | Var n => nth n l 0
  end.

Definition term2nat (l : list nat) (t : term) : nat :=
  match t with Succ n b => n + base2nat l b end.
\end{minted}
\end{frame}

\begin{frame}[fragile]\justifying
  \frametitle{Representation of Formulae}
  A formula is formed from an atom, or from other propositions by way of the typical connectives and quantifiers.
\begin{minted}{coq}
Inductive atom : Set := Equal : term -> term -> atom.

Inductive form : Set :=
| FAlse : form
| Atom : atom -> form
| Or : form -> form -> form
| And : form -> form -> form
| Imp : form -> form -> form
| Forall : form -> form
| Exists : form -> form.
\end{minted}
\end{frame}

\begin{frame}[fragile]
\begin{minted}{coq}
Definition atom2Prop (l : list nat) (a : atom) :=
  match a with
    Equal t1 t2 => term2nat l t1 = term2nat l t2
  end.

Fixpoint form2Prop (l : list nat) (f : form) : Prop :=
  match f with
  | FAlse => False
  | Atom a => atom2Prop l a
  | Or a b => form2Prop l a \/ form2Prop l b
  | And a b => form2Prop l a /\ form2Prop l b
  | Imp a b => form2Prop l a -> form2Prop l b
  | Forall a => forall x : nat, form2Prop (x :: l) a
  | Exists a => exists x : nat, form2Prop (x :: l) a
  end.
\end{minted}
\end{frame}
\begin{frame}
  \begin{center}
    {\Huge Thank you}
  \end{center}
\end{frame}

\end{document}
